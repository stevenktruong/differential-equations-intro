\subsection{Introduction}
In this section, we will be dealing with differential equations of the form $P\p{x, y} + Q\p{x, y}y' = 0$, which we will write as $P\p{x, y}\,\diff{x} + Q\p{x, y}\,\diff{y} = 0$.

Our goal is to find a real-valued function $F\p{x, y}$ such that $F\p{x, y} = C$ implicitly defines a solution to the differential equation. To illustrate this, consider the following example.

\textit{Example:} \\
Let $P\p{x, y} = x$, $Q\p{x, y} = y$. We claim that the equation
\begin{equation}
	F\p{x, y} = x^2 + y^2 = C	
\end{equation}
implicitly defines a solution to the exact differential equation
\[
	x\,\diff{x} + y\,\diff{y} = 0.
\]
Taking the derivative with respect to $x$ on both sides of (1) yields
\begin{align*}
	2x + 2y\frac{\diff{y}}{\diff{x}} &= 0 \\
	2x\,\diff{x} + 2y\,\diff{y} &= 0 \\
	x\,\diff{x} + y\,\diff{y} &= 0
\end{align*}
as we wanted to show.

So, $x^2 + y^2 = C$ are solutions to the problem, which are circles of radius $\sqrt{C}$ centered at the origin.

\begin{definition}
	Suppose that the solutions for the differential equation $P\p{x, y} + Q\p{x, y}y' = 0$ are implicitly given by $F\p{x, y} = C$. Then the set $\left\{\p{x, y} \in \mathbb{R}^2 \mid F\p{x, y} = C \right\}$ are called the \textbf{integral curves} of the differential equation.	
\end{definition}

